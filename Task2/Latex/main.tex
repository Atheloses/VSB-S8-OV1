\documentclass{article}% use option titlepage to get the title on a page of its own.
\usepackage{booktabs}
\usepackage{indentfirst}
\usepackage{listings}

\title{Operační výzkum I\\
    \large Task 2 - algebraic a tableau form}

\date{6.3.2021}
\author{Martin Pustka} 

\begin{document}

\maketitle
\newpage

\section{Task 2}

\subsection{Zadání}
\begin{tabular}{rrcc}
    $10x_1$     & + $20x_2$  & =      & $Z$   \\
    \midrule
    - $x_1$     & + $2x_2$   & $\leq$ & $15$  \\
    $x_1$       & + $x_2$    & $\leq$ & $12$  \\
    $5x_1$      & + $3x_2$   & $\leq$ & $45$  \\
    \midrule
    $x_1$ &  & $\geq$ & 0 \\
    &  $x_2$ & $\geq$ & $0$ \\
\end{tabular}

\newpage
\subsection{Algebraic}

\begin{tabular}{rrrrrcc}
    $10x_1$   & + $20x_2$  &          &          &          & = & $Z$   \\
    - $1x_1$  & + $2x_2$   & + $1x_3$ &          &          & = & $15$  \\
    $1x_1$    & + $1x_2$   &          & + $1x_4$ &          & = & $12$  \\
    $5x_1$    & + $3x_2$   &          &          & + $1x_5$ & = & $45$  \\
\end{tabular}
\newline
\textbf{
    (0,0,15,12,45) \\
    $x_2$ roste nejrychleji - není optimální řešení\\
}

\begin{tabular}{rrrcclrrcl}
    $1x_3$ & & & =   & $15$ & - $2x_2$ & $=>$ & $x_2$ & = & $15/2$ \\
    & $1x_4$ & & =   & $12$ & - $1x_2$ & $=>$ & $x_2$ & = & 12 \\
    & & $1x_5$ & =   & $45$ & - $3x_2$ & $=>$ & $x_2$ & = & 15 \\
\end{tabular}
\newline
\textbf{
    (0, 15/2, 0, 9/2, 45/2) \\
    $x_3$ - leaving variable \\
    $x_2 = (15 + x_1 - x_3)/2$ \\
}

\begin{tabular}{rrrrrcc}
    $10x_1$   & + $10(15 + x_1 - x_3)$   &          &          &          & = & $Z$   \\
    - $1x_1$  & + $2x_2$                 & + $1x_3$    &          &          & = & $15$  \\
    $1x_1$    & + $(15 + x_1 - x_3)/2$   &          & + $1x_4$ &          & = & $12$  \\
    $5x_1$    & + $3(15 + x_1 - x_3)/2$  &          &          & + $1x_5$ & = & $45$  \\
\end{tabular}
\newline
\textbf{
    provedení úprav \\
}

\begin{tabular}{rrrrrcc}
    $20x_1$  & + 150    & - $10x_3$ &          &          & = & $Z$   \\
    - $1x_1$ & + $2x_2$ & + $1x_3$  &          &          & = & $15$  \\
    $3x_1$   &          & - $1x_3$  & + $2x_4$ &          & = & $9$  \\
    $13x_1$  &          & - $3x_3$  &          & + $2x_5$ & = & $45$  \\
\end{tabular}
\newline
\textbf{
    $x_1$ roste nejrychleji - není optimální řešení\\
}

\begin{tabular}{rrcclrrcl}
    $2x_4$  &  & =   & $9$  & - $3x_1$  & $=>$ & $x_1$ & = & 3 \\
    & $2x_5$   & =   & $45$ & - $13x_1$ & $=>$ & $x_1$ & = & 45/13 \\
\end{tabular}
\newline
\textbf{
    (3, 9, 0, 0, 3) \\
    $x_4$ - leaving variable \\
    $x_1 = (9 + 1x_3 - 2x_4)/3$ \\
}

\begin{tabular}{rrrrrcc}
    $20(9 + 1x_3 - 2x_4)/3$  & + 150    & - $10x_3$ &          &          & = & $Z$   \\
    - $1(9 + 1x_3 - 2x_4)/3$ & + $2x_2$ & + $1x_3$  &          &          & = & $15$  \\
    $3x_1$                   &          & - $1x_3$  & + $2x_4$ &          & = & $9$  \\
    $13(9 + 1x_3 - 2x_4)/3$  &          &  - $3x_3$ &          & + $2x_5$ & = & $45$  \\
\end{tabular}
\newline
\textbf{
    provedení úprav \\
}

\begin{tabular}{rrrrrcc}
      210  &          & - $10x_3/3$ & - $40x_4/3$ &             & = & $Z$   \\
           & + $2x_2$ & + $2x_3/3$  & + $2x_4/3$  &             & = & $18$  \\
    $3x_1$ &          & - $1x_3$    & + $2x_4$    &             & = & $9$  \\
           &          &   $4x_3/3$  & - $26x_4/3$ & + $2x_5$    & = & $6$  \\
\end{tabular}
\newline

\begin{tabular}{lcr}
    Z     & =     & 210 \\
    $x_1$ & =     & 3 \\
    $x_2$ & =     & 9 \\
\end{tabular}

\newpage
\subsection{Tableau}
\begin{tabular}{rrrrrrrr}
            & Z     & $x_1$ & $x_2$ & $x_3$ & $x_4$ & $x_5$ &       \\
     Z      & 1     & -10   & -20   & 0     & 0     & 0     & 0     \\
     $x_3$  & 0     & -1    & 2     & 1     & 0     & 0     & 15    \\
     $x_4$  & 0     & 1     & 1     & 0     & 1     & 0     & 12    \\
     $x_5$  & 0     & 5     & 3     & 0     & 0     & 1     & 45    \\
\end{tabular}
\newline
\textbf{
    nejvíce v mínusu je $x_2$ \\
    $x_3$ má nejmenší hodnotu $15/2$ \\
}

\begin{tabular}{rrr|r|rrrr}
        & Z     & $x_1$ & $x_2$ & $x_3$ & $x_4$ & $x_5$ &       \\
    Z      & 2     & -40   & 0     & 20    & 0     & 0     & 300   \\
    \midrule
    $x_2$  & 0     & -1    & 2     & 1     & 0     & 0     & 15    \\
    \midrule
    $x_4$  & 0     & 3     & 0     & -1    & 2     & 0     & 9     \\
    $x_5$  & 0     & 13    & 0     & -3    & 0     & 2     & 45    \\
\end{tabular}
\newline
\textbf{
    nejvíce v mínusu je $x_1$ \\
    $x_4$ má nejmenší hodnotu $3$ \\
}

\begin{tabular}{rr|r|rrrrr}
           & Z     & $x_1$ & $x_2$ & $x_3$ & $x_4$ & $x_5$ &       \\
    Z      & 6     & 0     & 0     & 20    & 80    & 0     & 1260  \\
    $x_2$  & 0     & 0     & 6     & 2     & 2     & 0     & 54    \\
    \midrule
    $x_4$  & 0     & 3     & 0     & -1    & 2     & 0     & 9     \\
    \midrule
    $x_5$  & 0     & 0     & 0     & 4     & -26   & 6     & 18    \\
\end{tabular}
\newline
\textbf{
    žádné mínusové v $Z$ - optimální řešení \\
    provedení úprav \\
}

\begin{tabular}{rrrrrrrr}
           & Z     & $x_1$ & $x_2$ & $x_3$  & $x_4$ & $x_5$ &       \\
    Z      & 1     & 0     & 0     & $10/3$ & $40/3$& 0     & 210   \\
    $x_2$  & 0     & 0     & 1     & $1/3$  & $1/3$ & 0     & 9     \\
    $x_4$  & 0     & 1     & 0     & $-1/3$ & $2/3$ & 0     & 3     \\
    $x_5$  & 0     & 0     & 0     & 2      & -13   & 3     & 9     \\
\end{tabular}
\newline

\begin{tabular}{lcr}
    Z & =     & 210 \\
    $x_1$ & =     & 3 \\
    $x_2$ & =     & 9 \\
\end{tabular}

\newpage
\subsection{Python zdrojový kód}
\begin{lstlisting}[language=Python, showstringspaces=false]
    from scipy.optimize import linprog

    #Task 2
    c = [-10,-20]
    A = [[-1,2],[1,1],[5,3]]
    b = [15,12,45]
    x0_b = (0,None)
    x1_b = (0,None)

    res = linprog(c, A_ub=A, b_ub=b, bounds=[x0_b, x1_b])
    #print(res)
    print('Task 2: \n' +
        f'x1 = {round(res.x[0])}, ' +
        f'x2 = {round(res.x[1])}, ' +
        f'profit = {-round(res.fun)}')

\end{lstlisting}

\subsection{Python output}
\begin{lstlisting}
    Task 2:
    x1 = 3, x2 = 9, profit = 210
\end{lstlisting}

\end{document}